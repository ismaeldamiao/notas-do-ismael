\chapter{Introdução}

\section{Notação}

\section{As leis da termodinâmica}

\subsection{Lei zero da termodinâmica}
\subsection{Primeira lei da termodinâmica}

\begin{axioma}
   Existem uma função $U:\mathcal{S}\rightarrow\mathbb{R}$
   e um conjunto $\mathcal{E}\subset\mathcal{S}$ tais que
   dados $x_0\in\mathcal{E}$ e $x_f\in\mathcal{E}$
   existe um caminho diferenciável $\Gamma\in\mathcal{E}$
   que liga $x_0$ e $x_f$ tal que
   $$ U(x_f) - U(x_0) = \int_{\Gamma} \delta W. \qedhere $$
\end{axioma}

\lipsum[1]

\begin{definicao}
   Seja a 1-forma $\delta Q = dU - \delta W$,
   o calor transferido em um processo $\Gamma$ é
   $$ Q(\Gamma) = \int_{\Gamma} \delta Q. \qedhere $$
\end{definicao}

\lipsum[2]

\subsection{Segunda lei da termodinâmica}
\subsection{Terceira lei da termodinâmica}
