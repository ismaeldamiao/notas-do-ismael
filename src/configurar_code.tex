\RequirePackage{listings}

\RequirePackage{tikz, tikz-cd}      % Insercao de imagens configuraveis.

\usetikzlibrary{shapes.geometric, arrows}
\tikzstyle{startstop} = [rectangle, rounded corners, minimum width=3cm, minimum height=1cm,text centered, draw=black]
\tikzstyle{io} = [trapezium, trapezium left angle=70, trapezium right angle=110, minimum width=3cm, minimum height=1cm, text centered, draw=black]
\tikzstyle{process} = [rectangle, minimum width=3cm, minimum height=1cm, text centered, draw=black]
\tikzstyle{decision} = [diamond, minimum width=3cm, minimum height=1cm, text centered, draw=black]
\tikzstyle{conector} = [circle, minimum width=0.5cm, minimum height=0.5cm, text centered, draw=black]
\tikzstyle{arrow} = [thick,->,>=stealth]
\tikzstyle{line} = [draw, -latex']

% see https://latexcolor.com/
\definecolor{codegreen}{rgb}{0,0.6,0}
\definecolor{codegray}{rgb}{0.5,0.5,0.5}
\definecolor{amaranth}{rgb}{0.9, 0.17, 0.31}
\definecolor{theBlack}{gray}{0.0}
\definecolor{theWhite}{gray}{0.9}

\RequirePackage{caption}
\DeclareCaptionFont{white}{\color{white}}
\DeclareCaptionFormat{listing}{\hspace*{-0.4pt}\colorbox{gray}{\parbox{\textwidth}{#1#2#3}}}
\captionsetup[lstlisting]{format=listing,labelfont=white,textfont=white}

\renewcommand{\lstlistingname}{Código}

\lstset{
   backgroundcolor=\color{gray!5!white},   % choose the background color; you must add \usepackage{color} or \usepackage{xcolor}; should come as last argument
   basicstyle=\mdseries\ttfamily\footnotesize\color{black}, % the size of the fonts that are used for the code
   breakatwhitespace=false,         % sets if automatic breaks should only happen at whitespace
   breaklines=true,                 % sets automatic line breaking
   captionpos=t,                    % sets the caption-position to bottom
   columns=fixed,                   % Using fixed column width (for e.g. nice alignment)
   escapechar=µ,
   numbers=none,                    % Not use numbers
   keepspaces=true,                 % keeps spaces in text, useful for keeping indentation of code (possibly needs columns=flexible)
   showspaces=false,                % show spaces everywhere adding particular underscores; it overrides 'showstringspaces'
   showstringspaces=false,          % underline spaces within strings only
   showtabs=false,                  % show tabs within strings adding particular underscores
   tabsize=3, 	                     % sets default tabsize to 3 spaces
   %frame=single,	                  % adds a frame around the code
   %rulecolor=\color{black},         % if not set, the frame-color may be changed on line-breaks within not-black text (e.g. comments (green here))
}

\lstdefinestyle{c}{
   language=C, % the language of the code (can be overrided per snippet)
   commentstyle=\color{codegray}, % comment style
   keywordstyle={\color{codegreen}\bfseries},
   stringstyle=\color{amaranth} % string literal style
}

\lstdefinestyle{f90}{
   language=Fortran, % the language of the code (can be overrided per snippet)
   commentstyle=\color{codegray}, % comment style
   keywordstyle={\color{codegreen}\bfseries},
   stringstyle=\color{amaranth} % string literal style
}

\lstdefinestyle{java}{
   language=Java, % the language of the code (can be overrided per snippet)
   commentstyle=\color{codegray}, % comment style
   keywordstyle={\color{codegreen}\bfseries},
   stringstyle=\color{amaranth} % string literal style
}

\lstdefinestyle{py}{
   language=Python, % the language of the code (can be overrided per snippet)
   commentstyle=\color{codegray}, % comment style
   keywordstyle={\color{codegreen}\bfseries},
   stringstyle=\color{amaranth} % string literal style
}

\lstdefinestyle{gnuplot}{
   language=Gnuplot, % the language of the code (can be overrided per snippet)
   commentstyle=\color{codegray}, % comment style
   keywordstyle={\color{codegreen}\bfseries},
   stringstyle=\color{amaranth} % string literal style
}

\lstdefinestyle{bash}{
   language=bash, % the language of the code (can be overrided per snippet)
   commentstyle=\color{codegray}, % comment style
   keywordstyle={\color{codegreen}\bfseries},
   stringstyle=\color{amaranth}, % string literal style
   morekeywords={apt, pacman, yum, zypper, dnf, dns, mkdir, cp, configure, make, tar},
}

\lstdefinestyle{pseudo}{
   commentstyle=\color{codegray}, % comment style
   keywordstyle={\color{codegreen}\bfseries},
   stringstyle=\color{amaranth}, % string literal style
   morekeywords={inicio, fim, para, ate, subrotinas, se, senao, entao, funcao, inteiro, enquanto, gnuplot, logico, verdadeiro, subrotina, variavel, inicialize},
   morestring=[b]",
   morecomment={[l]//},
   inputencoding=utf8,
   extendedchars=true,
   literate=%
      {á}{{\'a}}1 {é}{{\'e}}1 {í}{{\'i}}1 {ó}{{\'o}}1 {ú}{{\'u}}1
      {Á}{{\'A}}1 {É}{{\'E}}1 {Í}{{\'I}}1 {Ó}{{\'O}}1 {Ú}{{\'U}}1
      {à}{{\`a}}1 {è}{{\`e}}1 {ì}{{\`i}}1 {ò}{{\`o}}1 {ù}{{\`u}}1
      {À}{{\`A}}1 {È}{{\'E}}1 {Ì}{{\`I}}1 {Ò}{{\`O}}1 {Ù}{{\`U}}1
      {ä}{{\"a}}1 {ë}{{\"e}}1 {ï}{{\"i}}1 {ö}{{\"o}}1 {ü}{{\"u}}1
      {Ä}{{\"A}}1 {Ë}{{\"E}}1 {Ï}{{\"I}}1 {Ö}{{\"O}}1 {Ü}{{\"U}}1
      {â}{{\^a}}1 {ê}{{\^e}}1 {î}{{\^i}}1 {ô}{{\^o}}1 {û}{{\^u}}1
      {Â}{{\^A}}1 {Ê}{{\^E}}1 {Î}{{\^I}}1 {Ô}{{\^O}}1 {Û}{{\^U}}1
      {ã}{{\~a}}1 {ẽ}{{\~e}}1 {ĩ}{{\~i}}1 {õ}{{\~o}}1 {ũ}{{\~u}}1
      {Ã}{{\~A}}1 {Ẽ}{{\~E}}1 {Ĩ}{{\~I}}1 {Õ}{{\~O}}1 {Ũ}{{\~U}}1
      {œ}{{\oe}}1 {Œ}{{\OE}}1 {æ}{{\ae}}1 {Æ}{{\AE}}1 {ß}{{\ss}}1
      {ű}{{\H{u}}}1 {Ű}{{\H{U}}}1 {ő}{{\H{o}}}1 {Ő}{{\H{O}}}1
      {ç}{{\c c}}1 {Ç}{{\c C}}1 {ø}{{\o}}1 {å}{{\r a}}1 {Å}{{\r A}}1
      {€}{{\euro}}1 {£}{{\pounds}}1 {«}{{\guillemotleft}}1
      {»}{{\guillemotright}}1 {ñ}{{\~n}}1 {Ñ}{{\~N}}1 {¿}{{?`}}1 {¡}{{!`}}1
      {←}{{$\leftarrow$}}1
      {!=}{{$\neq$}}1
}

\makeatletter
% \expandafter for the case that the filename is given in a command
\newcommand{\replunderscores}[1]{\expandafter\@repl@underscores#1_\relax}

\def\@repl@underscores#1_#2\relax{%
    \ifx \relax #2\relax
        % #2 is empty => finish
        #1%
    \else
        % #2 is not empty => underscore was contained, needs to be replaced
        #1%
        \textunderscore
        % continue replacing
        % #2 ends with an extra underscore so I don't need to add another one
        \@repl@underscores#2\relax
    \fi
}
\makeatother

\RequirePackage[most]{tcolorbox}
\RequirePackage{fontawesome5}       % Permite o uso da fonte Font Awesome

\newcommand{\inserircodigo}[2]{
   \begin{tcolorbox}[
      colback=gray!5!white,
      colframe=gray!75!black,
      title=\ttfamily\replunderscores{#2}
   ]
      \lstinputlisting[style = #1]{./SourceCodes/#2}
   \end{tcolorbox}
}
