\chapter{Introdução} % Adciona um capítulo cujo título é informado ente {}

% Note que você pode começar a adcionar seus texto e usar ascentos
% Considere que o texto não comentado também está te disendo algo
A Introdução é onde é feito um resumo histórico e teórico sobre o que
está sendo estudado.

\ref{eq:nomedaequacao} O Comando prescedente faz uma citação ao que está 
entre \{\} enquanto que o comando que se segue adcionada uma equação 
numerada:

\begin{equation}\label{eq:nomedaequacao} % Aqui é onde você dá um nome ao bloco, nesse
    % caso equação, para se referir a ele durante o texto com o \ref{}.
    \sum_{i=1}^{n} \vec F=\sum_{i=1}^{n} \frac{d\vec p}{dt}=\sum_{i=1}^{n} \ddot{\vec p}
\end{equation}

\noindent este comando serve para adcionar uma linha não indentada.

Confira em \url{https://pt.wikipedia.org/wiki/Ajuda:Guia_de_edi%C3%A7%C3%A3o/F%C3%B3rmulas_TeX}
um guia completo sobre os comandos para utilizar notação matemática em 
$\LaTeX$.

E sim $x=x_0cos(\omega_0t)$, dessa forma é possível adcionar equações 
no meio de uma linha simplesmente colocando a equação entre \$\$.

\begin{equation}\label{eq:part2}
    T=2\pi \sqrt{\frac{m}{k}}
\end{equation}

Os parágrafos da introdução precisam ter cidatas as fontes onde você 
aprendeu o que está dizendo. Este comando server pra 
isso.\cite{LoiElectricite}\cite{moyses1}

Utilizei os seguintes comandos para enunciar a primeira lei do movimento 
de Newton:

\begin{lei}
    Corpus omne perseverare in statu suo quiescendi vel movendi uniformiter in directum, nisi quatenus illud a viribus impressis cogitur statum suum mutare.\cite{principia1}
\end{lei}

Essa é uma citação da segunda lei do movimento
\footnote{Este comando server para adcionar uma nota de rodapé}
\footnote{Sir Isaac Newton}:

\begin{citacao}
    Lex II: Mutationes motis proportionalem esse vi motrici impressæ, et fieri secundum lineam retam qua vis illa imprimitur. - N. Isaac\cite{principia1}
\end{citacao}

Este comando serve para adcionar um teorema:

\begin{teorema}
    Sejam $\varphi_j : \mathbb{R} \rightarrow \mathbb{R}, j = 1, ..., n$\: $n$ funções de classe $C^n$, se o Wronskiano 
    associado a estas funções é não-nulo então ela são linearmente independentes.
\end{teorema}

Aproveito os comando de axioma para enunciar os três pilares da lógica:

\begin{axioma}Identidade\\
    Toda proposição é equivalente a si mesma.
\end{axioma}

\begin{axioma}Não contradição\\
    Toda proposição ou é falsa ou é verdadeira nunca os dois.
\end{axioma}

\begin{axioma}Identidade\\
    Toda proposição é exclusivamente ou falsa ou verdadeira nunca outro.
\end{axioma}

% -----------------
% SEçÃO
% -----------------
\section{Seção}

Isto é uma Seção.

\subsection{Sub-seção}

Isto é uma Sub-seção.

\subsubsection{Sub-sub-seção}

Isto é uma Sub-sub-seção.