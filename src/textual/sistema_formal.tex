\chapter{Sistema Formal}

A fim de comunicar o conhecimento deve-se necessariamente
fazer usa da linguagem.
Com o objetivo de melhor expressar os conceitos desejados
pode-se inclusive conceber uma linguagem com o fim de
expressar esses conceitos.

A linguagem que escolhi para o discurso desse livro tem suas bases
na lógica clássica,
para que possa gozar também do privilégio do cálculo
tal lógica será expressada (em partes) por meio de um sistema formal.
Esses serão os recursos linguísticos dos quais farei uso.

% %%%%%%%%%%%%
% SECTION
% %%%%%%%%%%%%
\section{Proposições}

O ponto de partida da lógica clássica é o estudo das proposições,
uma proposiçao é uma fórmula que pode ser verdadeira (V) ou falsa (F).
Tome por exemplo as proposições abaixo.

\begin{itemize}
   \item[$\varphi_1$:] Minha gata mia.
   \item[$\varphi_2$:] Essa equação é linear.
\end{itemize}

\noindent
A primeira proposição é verdadeira, o sei por experiência,
enquanto que a segunda sepende da equação que nos é desconhecida,
suponha que a equação seja a Lei de Faraday então a proposição é falsa.
Note que no vernáculo existem fórmulas que não são proposições,
por exemplo:

\begin{itemize}
   \item[\varphi_3] Onde está o gato?
   \item[\varphi_4] Resolva essa integral.
\end{itemize}

% %%%%%%%%%%%%
% SECTION
% %%%%%%%%%%%%
\section{Predicados}

% %%%%%%%%%%%%
% SECTION
% %%%%%%%%%%%%
\section{Quantificadores}

% %%%%%%%%%%%%
% SECTION
% %%%%%%%%%%%%
\section{Sistema Formal}

Nesta seção exponho as regras gramaticais da linguagem formal
que irei utilizar.
Utilizo a letra grega $\varphi$ (possivelmente com subindices)
para representar fórmulas na linguagem,
já a letra grega $\sigma$
(possivelmente com subindices ou em maiúsculo, $\Sigma$)
uso para representar simbolos no alfabeto que serão chamados de termos,
os termos se destinam a representar sujeitos, predicados e variáveis.
Reservo para o alfabeto o simbolo $\in$
para representar a relação de pertinência ou partícula preposicional,
reservo também o símbolo $\uparrow$ para representar
a operação de negação da conjunção
e o símbolo $\forall$ para representar o quantificador existencial,
outros símbolos são reservados mais adiante para abreviações.
A linguagem utiliza as seguintes regras gramaticais:

\begin{itemize}
   \item
      Se $\varphi_1$ e $\varphi_2$ forem fórmulas então
      $(\varphi_1\uparrow\varphi_2)$ também será uma fórmula.
   \item
      Se $\sigma$ e $\Sigma$ forem termos então $(\sigma\in\Sigma)$
      será uma fórmula.
   \item
      Se $\varphi$ for uma fórmula e $\sigma$ for um termo então
      $\forall_{\sigma}\varphi$ será uma fórmula.

   \item[$\uparrow\mathtt I$:] Introdução do $\uparrow$
   
      Infere-se $(\varphi_1\uparrow\varphi_2)$
      quando $\varphi_1$ pode ser inferido de $(\varphi_2\uparrow\varphi_2)$.

      $
         \{\{\varphi_1\} \vdash (\varphi_2\uparrow\varphi_2)\} \vdash
         (\varphi_1\uparrow\varphi_2)
      $

   \item[$\uparrow\mathtt E$:] Eliminação do $\uparrow$
   
      Infere-se $(\varphi_2\uparrow\varphi_2)$
      a partir de $\varphi_1$ e de $(\varphi_2\uparrow\varphi_1)$.

      $
         \{\varphi_1, (\varphi_2\uparrow\varphi_1)\} \vdash
         (\varphi_2\uparrow\varphi_2)
      $

   \item[$\uparrow\uparrow\mathtt E$:] Eliminação do duplo $\uparrow$

      Infere-se $\varphi_2$
      a partir de $\varphi_1$ e de
      $((\varphi_2\uparrow\varphi_2)\uparrow\varphi_1)$.

      $
         \{\varphi_1, ((\varphi_2\uparrow\varphi_2)\uparrow\varphi_1)\}\vdash
         \varphi_2
      $

   \item[$\exists\mathtt I$:] Generalização existencial

      Infere-se $\exists_{\sigma_2} (\sigma_2\in\Sigma)$
      a partir de $(\sigma_1\in\Sigma)$.

      $\{(\sigma_1\in\Sigma)\}\vdash \exists_{\sigma_2} (\sigma_2\in\Sigma)$

      \item[$\exists\mathtt E$] Instanciação existencial

      Infere-se $\varphi$ a partir de $\exists_{\sigma_2} (\sigma_2\in\Sigma)$
      quando $\varphi$ também pode ser inferido a partir de $(\sigma_1\in\Sigma)$.

      $
         \{\exists_{\sigma_1} (\sigma_1\in\Sigma),
         \{\sigma_2\in\Sigma\}\vdash\varphi\}\vdash\varphi
      $
\end{itemize}
