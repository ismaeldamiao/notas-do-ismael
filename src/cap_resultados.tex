\chapter{Resultados e análise}
  \vspace{0.5cm}

  Mostre tabelas e gráficos com os dados.

  Organize esses dados de tal forma que eles informem algo, aqui os dados serão expostos de tal forma que mais adiante seja possível tirar conclusões deles.

  Com o \LaTeX  podemos adcionar tabelas:

  \begin{table}[H]
     \centering
     \caption{\label{tab:part1}\footnotesize Título}
     \begin{tabular}{c|cccc}
        Haste    & $t_1(s)$ & $t_2(s)$ & $t_3(s)$ & $t_4(s)$ \\ \hline
        Cobre    & 41       & 74       & 99       & 135      \\
        Alumínio & 35       & 57       & 89       & 129      \\
        Latão    & 56       & 107      & 181      & 239     
     \end{tabular}
  \end{table}

  \noindent confira \url{https://tablesgenerator.com/}.

  Com o \LaTeX  podemos adcionar figuras:

  \begin{figure}[H]
     \centering
     \caption{\label{fig:coordenadas}\footnotesize Título.}
     \includegraphics[scale=0.2]{img_ismael}
     \legend{\footnotesize Fonte:}
  \end{figure}

  \noindent ou

  \begin{figure}[H]
     \centering

     \begin{subfigure}[t]{0.4\textwidth}%
        \centering
        \includegraphics[scale=0.1]{img_ismael}
        \caption{\label{fig:placas}\footnotesize Exemplo de duas figuras.}
     \end{subfigure}
     \hspace{0.1cm}
     \begin{subfigure}[t]{0.4\textwidth}%
        \centering
        \includegraphics[scale=0.1]{img_ismael}
        \caption{\label{fig:pontos}\footnotesize Numa só.}
     \end{subfigure}

     \legend{\footnotesize Fonte: \url{http://ismaeldamiao.blogspot.com} - Acesso em 20 de nov. 2019.}
  \end{figure}

  Com o \LaTeX  podemos adcionar gráficos:

  \begin{figure}[H]
     \centering
     \caption{\label{fig:hooke}\footnotesize Exemplo de gráfico com GNUPlot.}
     \begin{gnuplot}[terminal=pdf]
        set key box top left
        data = "data_1.dat"

        set ylabel "Força (N)"
        set xlabel "Deslocamento (m)"

        f(x) = A * x + B

        fit f(x) data via A, B

        plot data lc rgb "blue" pt 4 ps .9 title "Dados", \
        f(x) w lp pt 7 ps 0.1 lc rgb "red" title "Ajuste linear"
     \end{gnuplot}
  \end{figure}

  Com o \LaTeX  podemos adcionar estruturas químicas:

  \begin{figure}[H]
     \centering
     \caption{\label{fig:polipropileno}\footnotesize Exemplo de estrutura química ($[C_3H_6]_n$).\cite{polipropileno}}
     \chemfig{
        \vphantom{CH_2}-[@{op,.75}]CH_2-CH(-[6]CH_3)-[@{cl,0.25}]
     }
     \makebraces[5pt,25pt]{\!\!\!n}{op}{cl}
     \bigskip
  \end{figure}